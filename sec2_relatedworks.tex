\section{Related Works}
\label{sec:related_works}
Related works can be grouped into three categories: \emph{co-movement patterns},
\emph{dynamic movement patterns} and \emph{trajectory mining frameworks}.
%The \emph{co-movement patterns} in literature consist 
%of five members, namely \emph{group}~\cite{wang2006grouppattern}, \emph{flock}~\cite{gudmundsson2004flock},
%\emph{convoy}~\cite{jeung2008convoy}, \emph{swarm}~\cite{li2010swarm} and \emph{platoon}~\cite{li2015platoon}.
%We have demonstrated the semantics of these patterns in Table~\ref{tbl:existing_co_patterns} and Figure~\ref{fig:related_work}. 
%In this section, we distinguish the parallel GCMP mining from these concepts.
%In this section, we focus on comparing the techniques used in these works.
%For more trajectory patterns other than \emph{co-movement patterns}, 
%interested readers may move to~\cite{zheng2015survey} for a comprehensive survey.


\subsection{Flock and Convoy}
The difference between \emph{flock} and \emph{convoy} lies 
in the object clustering methods. In \emph{flock},
objects are clustered based on their distances. Specifically, the
objects in the same cluster need to have a pairwise distance less than \emph{min\_dist}. 
This essentially requires the objects to be within a disk-region of delimiter less than \emph{min\_dist}.
In contrast, \emph{convoy} clusters objects using density-based spatial clustering~\cite{ester1996density}.
Technically, \emph{flock} utilizes a $m^{th}$-order Voronoi diagram~\cite{laube2005finding} to detect whether
a subset of $n$ ($n \geq m$) objects stay in a disk region.
%a subset of object with size greater than $m$ stays in a disk-region.
\emph{Convoy} employs
a trajectory simplification~\cite{douglas1973linesimplification} technique to boost pairwise distance computations in
the density-based clustering.
After clustering, both \emph{flock} and \emph{convoy} use a sequential scanning
method to examine each snapshot. 
During the scan, the object groups that do 
not appear in consecutive snapshots are pruned.
%During the scan, object
%groups that appear in consecutive timestamps are detected. Meanwhile, the object groups that do not
%match the consecutive constraint are pruned. 
However, such a method faces high complexity when supporting other patterns.
For instance, in \emph{swarm}, the candidate set during the sequential scanning grows
exponentially, and many candidates can only be pruned after the entire dataset are scanned.

\subsection{Group, Swarm and Platoon}
Different from \emph{flock} and \emph{convoy}, all the \emph{group},\emph{swarm} and \emph{platoon}
patterns have more relaxed constraints on the pattern duration. Therefore, their techniques of mining are of
the same skeleton. The main idea of mining is to grow object set from an empty set
in a depth-first manner. During the growth, various pruning techniques are provided to prune 
unnecessary branches. \emph{Group} pattern uses a VG-graph to guide the pruning of false candidates~\cite{wang2006grouppattern}.
\emph{Swarm} designs two more pruning rules called backward pruning and forward pruning~\cite{li2010swarm}. \emph{Platoon}
further leverages a prefix table structure to steer the depth-first search, which shows efficiency 
as compared to the other two methods.
%As shown by Li et.al.~\cite{li2015platoon},
%\emph{platoon} outperforms other two methods in efficiency. 
%However, the three patterns are not able to directly discover GCMPs.
However, the pruning rules adopted by the three patterns heavily rely on depth-first search which lose efficiency in parallel scenario.
%Furthermore, their pruning rules heavily rely on the depth-first search nature, which lose their efficiencies
%in the parallel scenario.

\subsection{Other Related Trajectory Patterns}
A closely related literature to co-movement patterns is the \emph{dynamic movement} patterns. Instead of requiring the same set of object traveling together, \emph{dynamic movement} patterns allow objects to temporally join or leave a group. Typical works include \emph{moving clusters}~\cite{kalnis2005movingclusters}, \emph{evolving convoy}~\cite{aung2010discovery}, \emph{gathering}~\cite{zheng2013gathering} etc. These works cannot model GCMP since they enforce global consecutiveness on the 
temporal domain.
%timestamps of a pattern. 

%\revised{Mining trajectory patterns on road networks has also been studied~\cite{li2007traffic,lee2011mining}. Since they leverage the domain knowledge of road networks,
%these models cannot be easily extended to broader scenarios such as visitor trajectories in shopping malls, people movements in cities and user check-in histories in social networks. 
%} 

%Another related literature on movement patterns is the pattern detection in road networks~\cite{}. However, these patterns require meta-information of road networks which  
%restricts the application scenarios such as visitors' trajectories in shopping mall and passengers' movements tracked by mobile devices. Nevertheless, GCMP is general enough for a broader application scenario which not only includes road networks, but also visitor trajectory, passenger movements and eye movements etc. }
 
\subsection{Trajectory Mining Frameworks}
Jinno et al. in~\cite{jinno2012paralleltpattern} designed a MapReduce based algorithm to efficiently support \emph{T}-pattern discovery, where a \emph{T}-pattern is a set of objects visiting the same place at simliar time. Li et al. proposed a framework of processing online \emph{evolving group} pattern~\cite{li2013onlinegroup}, which focuses on supporting efficient updates of arriving objects. 
As these works essentially differ from co-movement pattern, their techniques cannot be directly applied to discover GCMPs.


%\revised{
%\subsection{Frequent Sequential Pattern  Mining}
%Mining Gap-restricted Sequential Patterns (GSP)~\cite{srikant1996mining,miliaraki2013mind}
%aims to find sequences which
%satisfy a gap constraint $G$ and occur more than $M$ times. 
%However, traditional GSP cannot support GCMP since they do not
%consider the closeness of objects, which is essential for trajectory patterns.
%%Thus, wemismatches motivates us to design the native GCMP for trajectory data.
%}
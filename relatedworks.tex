\section{Related Works}
\label{sec:related_works}
There are many works has been done on patterns mining in trajectory databases. We loosely classify these patterns into \emph{General Trajectory Pattern}s and \emph{Co-movement Pattern}s. The difference is that \emph{Co-movement Pattern}s require all members of a cluster stay together during the pattern lifetime. We summarize the related works as follows:

\subsection{General Trajectory Pattern}
General Trajectory pattern mining a hot field in trajectory analysis. Previous works
define various patterns~\cite{
kalnis2005movingclusters,
li2010periodicpattern,zheng2013gathering,jinno2012paralleltpattern,li2013onlinegroup} over trajectory data, which have proven their usefulness under different applications~\cite{giannotti2007survey}. 

Kalnis et al. proposed \emph{moving clusters} pattern~\cite{kalnis2005movingclusters}. In such a pattern, objects form clusters at each snapshot. For consecutive snapshots, the clusters in the pattern should have a Jaccard index greater than a threshold. Under such a scheme, the difference between cluster members in snapshots accumulates, therefore the clusters at later snapshot may be very different from those in previous snapshots. The online extension is studied by Li et al.~\cite{li2013onlinegroup}.

Li et al. considered the \emph{periodic} pattern in~\cite{li2010periodicpattern}. This pattern mines the objects with periodic behaviors. An example of such a pattern is that individuals go to work together every day. 
It is commented in~\cite{giannotti2007survey} that \emph{periodic} pattern is unsuitable for discovering movements, since it is unreasonable to expect an object to repeat its behavior exactly during each time period considered.

Zhang et al. proposed the \emph{gathering} pattern in \cite{zheng2013gathering}. It is similar to \emph{flock} pattern~\cite{gudmundsson2004flock} but with the relaxation on the members of clusters. Instead of fixing the members in clusters as in~\cite{gudmundsson2004flock}, \emph{gathering} pattern allows members in clusters leave and join during the pattern duration. Since it relaxes the member constraints, it is unable to model co-movement patterns.


\subsection{Co-movement Pattern}
The work most related to ours is those on \emph{co-movement} patterns. We summarize the typical patterns as follows:
\subsubsection{group}
Wang et al. defined \emph{group pattern}~\cite{wang2006grouppattern}, which aims to find the set of objects travelling together at certain time intervals. In \emph{group pattern}, groups at each snapshot is identified by a disc-based clustering method, where each cluster forms a circle within a radius. It is argued in later works~\cite{jeung2008convoy,li2010swarm} that such disc-based clustering is not effective as \emph{density}-based clustering where the later one may find clusters of arbitrary shapes.

\subsubsection{flock}
Gudmunsson et al. proposed \emph{flock} pattern in 
\cite{gudmundsson2004flock,gudmundsson2006flock} and Vieria et al. followed up with an online version in~\cite{
vieira2009onlineflock}. A \emph{flock} pattern tries to find the set of objects that stay in a circular ranged cluster for a minimum duration. Such a pattern is useful in detecting the moving companions. However, similar as \emph{group pattern}, it uses disc-based clustering, which suffers the same deficiency in discovering arbitrary shaped clusters. \emph{Flock} pattern has many derivatives. In \cite{benkert2006meet}, Benkert et al. studied \emph{meet} pattern, which require the clusters in the pattern to be geographically stationary. Giannotti et al. studied \emph{leadership} pattern~\cite{andersson2007leadership} which requires a leader object exists for each flock cluster.
%In \cite{benkert2006meet}, Benkert et al. studied \emph{meet} pattern. A \emph{meet} pattern aims to find a set of objects stay
%stationary with in a circular range for some durations. This pattern does not consider the temporal movement of objects. Giannotti et al. studied \emph{leadership} pattern~\cite{andersson2007leadership} which requires a set of objects stay relatively within a circular range at each snapshots for some durations and there is at least one object is heading (leader). It is shown in~\cite{giannotti2007survey} that both \emph{Meet} and \emph{leadership} patterns are special cases of the \emph{flock} pattern~\cite{gudmundsson2004flock}.



\subsubsection{convoy}
Jeung et al. proposed \emph{convoy} pattern that extends \emph{flock} pattern by replacing the disc-based clustering with \emph{density}-based clustering. Such an relaxation brings a high complexity of repeatedly running DBSCAN~\cite{birant2007st} at every snapshot. To reduced the complexity, Jeung et al. designed a filter-refine approach which first uses simplification technique~\cite{douglas1973linesimplification} to filter far away objects, and then uses coherent moving method~\cite{kalnis2005movingclusters} to find the exact convoy patterns. Along with \emph{convoy} pattern, Aung et al. proposed \emph{dynamic convoy} and \emph{evolving convoy} patterns. In \emph{dynamic convoy}, the cluster members are allowed to be absent briefly during the convoy lifetime, while \emph{evolving convoy} allows the convoy to grow or shrink in cardinality during the life time. Tang et al. also addresses the online extension in~\cite{tang2012onlineconvoy}.
\subsubsection{swarm}
The major argument on \emph{convoy} pattern is that \emph{convoy} requires the consecutiveness in the lifetime, which may lose many interesting patterns. To remedy, Li et al. proposed the \emph{swarm} pattern~\cite{li2010swarm} which completely relaxes the consecutiveness in \emph{convoy}. In \emph{swarm}, objects can collectively leave the cluster for a long time and then join back in later time. The only requirement in \emph{swarm} is that each member in the cluster needs to accumulate to a certain duration. In~\cite{li2010swarm}, the authors proposed a depth-first search based pruning algorithm to efficiently discover \emph{swarm} patterns.
\subsubsection{platoon}
Recently Li et al. argued that \emph{swarm} is to loose in the temporal consecutiveness and proposed \emph{platoon} pattern in~\cite{li2015platoon}. In \emph{platoon} pattern, the clusters should lasts for at least a certain during before dismiss. Meanwhile, \emph{platoon} allow the clusters to form again at future times. Li et al. demonstrated the such extension is more general and can support swarm and convoy patterns by setting appropriate parameters. Li et al. also provide a similar depth-first search approach as in~\cite{li2010swarm}. In addition, they adapted a prefix pruning method to further improve efficiency. It is notable that in both \cite{li2010swarm} and \cite{li2015platoon}, authors consider the input to be the clusters at each snapshot, which ignores the clustering time.

\subsection{Pattern Mining Frameworks}
Jinno et al. recently studied the problem of processing \emph{T}- pattern~\cite{giannotti2007survey} in parallel platform \cite{jinno2012paralleltpattern}. A \emph{T}-pattern discovers a set of objects visiting the same the place in a close time interval. Such a pattern differs from moving object pattern in that \emph{T}-pattern does not consider the movement of objects. Jinno et al. in~\cite{jinno2012paralleltpattern} designed a MapReduce based algorithm for efficiently support \emph{T}-pattern discovery. However, as the nature of differences between the patterns, their work cannot directly applied on the co-moving object pattern discovery. Li et al. recently proposed a framework of processing online \emph{evolving group} pattern~\cite{li2013onlinegroup}. The \emph{evolving group} is similar to \emph{moving cluster} pattern with focus on the member updates in clusters, which is different with \emph{co-movement} pattern. Moreover the framework is developed for centralized system, thus is different with our work.

\subsection{Trajectory Clustering}
Another field that is related to our work is trajectory clustering~\cite{he2011mrdbscan,lee2007partitionandgroup,
li2004clusteringmovingobjects}. Lee et al. proposed a partition and group algorithm in~\cite{lee2007partitionandgroup}. Their clustering method does not consider the temporal constraint and groups trajectories from different time points together. Li et al. proposed a \emph{micro-clustering} technique~\cite{li2004clusteringmovingobjects} for cluster moving objects based on their moving directions. However, such a technique clusters moving objects based on entire trajectories, thus is not able to identify patterns that are at local trajectories. He et al. proposed a parallel DBSCAN algorithm in~\cite{he2011mrdbscan} running on MapReduce platform. Their method uses a four-stage process to decouple the dependencies in DBSCAN. However their method only focuses on DBSCAN for one datasets, where exploiting the relationship between multiple DBSCANs remains unexplored.
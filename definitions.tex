\section{Definitions}
\label{sec:definition}
Let $\mathbb{O} = \{o_1 ,o_2,...,o_n\}$ denote the set of objects in concern. 
We use $T \subseteq \{1,2,...,m\}$ to denote a descritized sequence of timestamps, where
$T[i]$ denotes the $i^{th}$ entry in the sequence and $|T|$ is the 
cardinality of the sequence.

A time sequence is consecutive if the difference between any of its neighboring entries
is 1, that is $\forall T[i], T[i+1] \in T, T[i+1] = T[i] + 1$. For any time sequence,
it can be decomposed into at least one consecutive portion(s). Therefore we can define the 
$L$-consecutiveness as follows:
%
%
%We say a sequence $T$ is \emph{(fully)consecutive}
%if and only if $\forall T[i] \in T, T[i+1] \in T$, $T[i+1] = T[i] + 1$. A subsequence $T^m$ is \emph{maximally consecutive}~\cite{li2015platoon} wrt. $T$ if and only if $T^m \subseteq T$ and $\nexists T^{m'}, T^m \subseteq T^{m'} \wedge T^{m'} $ is consecutive. For example, let $T=\{1,2,3,5,6\}$, then $T_1=\{1,2,3\}$ and $T_2=\{5,6\}$ are two maximally consecutive subsequences wrt. $T$. We then define the $L$-consecutiveness as follows:

\begin{definition}[$L$-consecutive~\cite{li2015platoon}]
A time sequence $T$ is $L$-consecutive if each of its consecutive portions 
has cardinality greater or equal to $L$.
\end{definition}

Take $T=\{1,2,3,5,6\}$ as an example. $T$ is not $3$-consecutive since $\{5,6\}$ has a cardinality less than $3$. $T$ is $2$-consecutive since both of its consecutive portions are of sizes at least $2$.
To control the closeness of timestamps, we further define the $G$-connected of a time sequence as follows:

\begin{definition}[$G$-connected]
A time sequence $T$ is $G$-connected if the gap between any of its neighboring timestamps is no greater than $G$. That is
 $\forall T[i],T[i+1] \in T, T[i+1]-T[i] \leq G$.
\end{definition}

Take $T=\{1,2,3,5,6\}$ as an example. $T$ is $2$-connected since the maximum gap between its neighboring time stamps is $5-3=2$. However, $T$ is not $1$-connected. Indeed, $1$-connection infers that $T$ is consecutive.

Given a timestamp $t$, objects with their locations at $t$ collectively form a \emph{snapshot}\footnote{Missing time stamps can be interpreted using existing methods such as linear interpolation~\cite{jeung2008convoy}.}.Objects in a snapshot can then be clustered based on the
closeness of their locations. Let $C_t(o_i)$ be the cluster which $o_i$ belongs to at time $t$, a general co-movement pattern can be defined as:
\begin{definition}[General Co-Movement Pattern]
A \emph{General co-movement pattern} $GCMP(M,K,L,G)= \langle O,T \rangle$ is a pair containing an object set $O\subseteq \mathbb{O}$ and a time sequence $T$, with the following constraints: (1)Closeness: $\forall o_i,o_j \in O, \forall t \in T, C_t(o_i) = C_t(o_j)$. (2) Significance: $|O| \geq M$. (3) Duration: $|T| \geq K$. (4) Consecutiveness: $T$ is $L$-consecutive, and (5)Connection: $T$ is $G$-connected.
%\begin{enumerate}
%\item{Closeness: $\forall o_i,o_j \in O, \forall t \in T, C_t(o_i) = C_t(o_j)$}
%\item{Significance: $|O| \geq M$}
%\item{Duration: $|T| \geq K$}
%\item{Consecutiveness: $T$ is $L$-consecutive}
%\item{Separateness: $T$ is $G$-separated}
%\end{enumerate} 
\end{definition}

The general co-movement pattern retains the patterns that discovered by all 
existing techniques (group, flock, convoy, swarm and platoon). 
The relationships between general co-movement pattern and other patterns are summarized in Table~\ref{tbl:patterns}.
\begin{table}
\centering
\begin{tabular}{|c|c|c|c|c|}
\hline 
Pattern & $L$ & $G$ & $M$ & $K$ \\ 
\hline
Group & $\cdot$ & $|T|$ & $2$ & $1$ \\
\hline
Flock & $K$ & 1 & $\cdot$ & $\cdot$ \\
\hline 
Convoy & $K$ & $1$ & $\cdot$ & $\cdot$\\ 
\hline 
Swarm & $1$ & $|T|$ & $\cdot$ & $\cdot$ \\ 
\hline 
Platoon & $\cdot$ & $|T|$ & $\cdot$ & $\cdot$\\ 
\hline 
\end{tabular} 
\caption{Representing other patterns using GCMP. $\cdot$ means user specified value.}
\label{tbl:patterns}
\end{table}

In particular, by setting $G=|T|$, we achieve the \emph{platoon} pattern. By setting $G=|T|,L=1$, we achieve the \emph{swarm} pattern. By setting $G=|T|$, $M=2$, $K=1$, we gain the \emph{group} pattern. Finally by setting $G=1$, we achieve the \emph{convoy} and \emph{flock} pattern. 
In addition to covering existing patterns, the general co-movement pattern avoids the \emph{loose connection} problem in \emph{platoon} pattern. As suggested previously, $\{1,2, 100,101\}$ will be included in the platoon pattern, however since they're too far away, this pattern is not prominent. By setting appropriate $G$, we are able to prune this anomaly. It is notable that GCMP is not able to be modeled by existing patterns.
 
It is also observable that the number of patterns in GCMP is exponential. To control the size of output, 
we notice that, for two patterns $P_1,P_2$, if $P_1.O \subseteq P_2.O$ and $P_2$ is a proper pattern, then $P_1$ is also a proper pattern. Therefore, we can define the \emph{Closed General Co-Movement Pattern} as follows:

\begin{definition}[Closed General Co-Movement Pattern]
A general co-moving pattern $P=\langle O, T \rangle$ is closed if and only if there does not exist another general co-moving pattern $P'$ s.t. $P.O \subseteq P'.O$.
\end{definition}

For example, let $n=2,k=2,l=1,g=1$, the pattern $\{o_1,o_2\}\{1,2,3,4\}$ is not a closed pattern, while $\{o_1,o_2,o_3\}$ $\{1,2,3,4\}$ is a closed pattern. The closed pattern avoids outputting duplicate information, thus making the result patterns more compact. 

LETS KEEP THE CLOSENESS INFORMATION AT THE MOMENT. IF NO CLOSENESS IS DEFINED, WE CANNOT REDUCE GCMP TO OTHER PATTERNS SINCE THOSE PATTERNS ARE ALL DEFINED AS ``CLOSED''
%For example, let $n=2,k=2,l=1,g=1$, the pattern $\{o_3,o_4\} \{1,2,3\}$ in Figure~\ref{fig:related_work} is not a closed pattern, while $\{o_3, o_4\} \{t_1,t_2,t_3\}$ is a closed pattern. 

%Although the general co-moving pattern is free from the clustering methods used at each snapshot, as suggested in~\cite{jeung2008convoy}, the \emph{density}-based clustering method is better in detecting object clusters with arbitrary spatio-shapes. Therefore in this paper, we mainly consider density-based clustering.
Our definition of GCMP is free from clustering method. Users are able to supply different clustering method to facilitate different needs. We currently expose both disk-region based clustering and DBSCAN as APIs to the user.

In summary, the goal of this paper is to present a parallel solution for discovering closed GCMP from large-scale trajectory data.

Before we move on to the algorithmic part, we list the notations that are used in the following sections.

\begin{table}[h]
\centering
\begin{tabular}{|c|c|} 
\hline
Symbols & Meanings \\
\hline 
$Tr_i$ & Trajectory of object $i$\\ 
\hline
$S_t$ & Snapshot of objects at time $t$ \\
\hline 
$\mathbb{O}$ & Set of objects \\ 
\hline 
$T$ & Time sequence \\
\hline
$C_t(o)$ & the cluster of object $o$ at time $t$ \\
\hline 
$Sr_i $ &  The star structure of object $i$ \\
\hline 
\end{tabular} 
\caption{Notions that will be used}
\end{table}
\section{Responses to Reviewer 3}
\emph{Though the implementation references Spark as the implementation
platform (as it also reflects in github repo provided), the algorithm design is
mostly limited to MapReduce, aka only Hadoop, which is a very small subset of
Spark. This may have a negative impact on the baseline implementation.
Particularly, recent releases of Spark have introduced window functions that can
be applied directly in the sliding window scenario here. Certainly, the algorithm
has to be redesigned to use DataFrame (and/or Spark SQL) interface, it has
been noted that this is a very efficient way to execute window functions in
Spark}

\response{
We agree that leveraging more advanced Spark features
could further improve the performance of our solutions. 
We do not heavily leverage these features because 
we wish to focus on designing parallel GCMP mining algorithms.
%to provide a general parallel framework which does not tie to Spark.
Nevertheless, we are in progress in developing
a better open-sourced version with Spark-tailored optimization. 

In the revision, we try to add a DataFrame based TRPM solution 
which aims to leverage the window functions from Spark-SQL.
However, to the best of our efforts, we could not achieve this goal in a short time. 
The major challenge is that current Spark do not support User Defined Aggregate Function (UDAF)
in window functions~\footnote{JIRA Spakr-8641: Native Spark Window Functions \url{https://issues.apache.org/jira/browse/SPARK-8641}}. Without UDAF, implementing
our Line sweep algorithm (Algorithm 1) using Spark primitives (e.g., sum, avg, rank and nth-tile)
is nontrivial. Nonetheless, we expect less performance
boost from TRPM even with the UDAF support. This is because the Line sweep algorithm is not
properly reducible, where Spark system would fail to leverage the partial aggregates
across multiple windows. As a result, it would be equivalent to our current TRPM
solution where windows are processed independently in parallel.
%TRPM although can be modeled using window functions over data frame, the aggregates
%it uses (i.e., the Line Sweeping method in Algorithm 1) cannot be simply represented
%by Spark primitives (e.g., sum, avg, rank, nth-tile). We then resort to
%the Spark User Defined Aggregate Function (UDAF). However, in Spark 1.5.2, the UDAF
%does not support window function syntax. Even in the latest Spark 1.6.0 (released 
%last month), UDAF does not fit with window function. We confirm these information
%with Spark development team~\footnote{In Spark JIRA tickets, fully supporting UDAF is still under development}.
%
%Nevertheless, since our Algorithm 1 is not reducible, even with UDAF, TRPM could
%not boost too much 
%
%we expect
%even with UDAF support TRPM would not boost too much.  Spark system is 
%%hard to leverage the partial aggregates acrocess
%
%we can analytically expect even using UDAF, TRPM would
%not boost too much. This is because our Algorithm 1 is not reducible. That
%is we cannot merge or share partial aggregates across different windows. In summary,
%We wish to convince the reviewer that current TRPM implementation is reasonable. 
}

\emph{In particular, it would be great to provide the
difference in the number of partitions/splits, the amount of processing and
memory usage (i.e., vcore and memory seconds) between TRPM and SPARE}

\response{We add Table 7 to include vcore-seconds 
and RDD memory usage for both TRPM and SPARE. We also add a clear description on partitions in Section 6.
}


\emph{A plot that breaks down the performance gain by each method would
be greatly appreciated by the readers.}

\response{We complement Figure 8 with the breakdown cost of TRPM. This
comparison showcases the benefits of both Star Partition (in mapshufle phase)
and Apriori Enumeration (in reduce phase) of SPARE.
}

\emph{Some choices of words may need to be reconsidered: for example, "a bunch
of" might not be appropriate in a technical paper.}

\response{We thank the reviewer for the correction. We have changed many unprofessional terms.
}


\emph{References to star partitioning and apriori pruning are missing. Though these
are well known, they need to clearly cited. At least the following reference is
missing:}

\response{We add the corresponding references in Section 5.1 and Section 5.2.}

\emph{In "In contrast, when utilizing the multicore
environment, SPAREP achieves 7 times speedup and SPARES achieves 10 times speedup.", was "multicore"
referring to the use of all 16 cores in one of your node? The specification of the machine was not clear.}

\response{ We use all cores in a single node to conduct the experiments. We revise the description in Section 6.2.3.
} 


\emph{The computation of "eta" was slightly different than that in the paper}

\response{
We have updated the GitHub repository to rectify the typo in the equation.}
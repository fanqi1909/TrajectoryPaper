\section{Response to Meta Reviewer}

\emph{Novelty: The GCMP generalization is not particularly novel. Please elaborate on the novelty of the work.}

\response{In addition to the points in Response 2, another novelty of our work lies in the 
techniques. 
Although the idea of star partition and apriori enumeration are well-known, 
linking them together to solve the problem in trajectory domain has not been attempted before.
Besides, we input heavy details on these algorithms (e.g., theoretical bounds of partition and
anti-monotonicity) which have not been previously proposed.
}

\emph{Parameter setting: Are we really interested in all sets of movements beyond a cardinality of size M? Please elaborate
on the motivation.}

\response {Similar responses with Response 3. }

\emph{Spark implementation: the paper references Spark as the implementation platform but the algorithm is limited to MapReduce. Spark has capabilities beyond MapReduce such as window functions. Please provide an implementation that uses the relevant features of Spark, or a convincing discussion of how these features can be useful and why they were not used.}

\response{Similar responses with Response 1 and 5. }

\emph{More details in the performance evaluation: Please provide more details about data partitioning and the effect of skew. Also provide details about how star partitioning and a priori pruning contribute to performance. Please provide references to these two methods.}

\response { 
Similar responses with Response 6, 7 and 9.
}
\section{Response to Review 2}

\emph{The GCMP generalization is not particularly novel. Putting a maximum gap size
on consecutive segments is well-known
in sequence mining published more than
10 years ago.}

\response{
We agree the fact that the gap parameter is an old concept. Nevertheless, our major contribution in
modeling lies in unifying existing movement patterns in the trajectory domain. 
The gap parameter is naturally introduced to 
solve the loose-connection anomaly (as in Figure 2).
%which is suffered by the state-of-the-art generalization (i.e., Platoon pattern). 
In the revision, we add Section 2.5 to accredit the origin of the \emph{gap} concept
and distinguish the sequential mining problem with our problem. 
}

\emph{In fact, I have doubts about formulating the GCMP patterns as proposed. Are
we really interested in all sets of movements beyond a cardinality of size $M$? 
Take the Taxi dataset as an example. Let say that there are lots of taxis going
from the airport to downtown. Let say that there are 1000 such taxis. For a
given $M$, are we interested in ${1000 \choose M}$ answers? 
So this speaks to the problem of picking $M$. If $M$ is 500, what is ${1000 \choose M}$? In fact, even if the
system gives the single answer of ${1000\choose 1000}$,
I am not sure I am
interested in this pattern as I already know that there are many taxis going from
the airport to downtown. What I think I am really interested in are GCMP that
are ``anomalous'', which is much harder to define.}

\response{
We agree that if no other parameters are given, the maximum number
of patterns discovered could be as large as ${ \mathbb{O} \choose M}$, which may
be less interesting to users.
However, such a case rarely occurs due to two reasons. First,
in real trajectory mining, parameter $M$
is associated with temporal parameters $K,L,G$ and they jointly prune many false
patterns. Second, the nature of our solutions prefer 
a larger pattern when possible~\footnote{TRPM shrinks an invalid larger pattern until it is valid. SPARE grows a valid smaller pattern until it is about to be invalid. Therefore both algorithms tend to find larger patterns.}.
For example, if $(o_1,o_2,o_3)$ is 
a pattern and $M = 2$, 
then both TRPM and SPARE only output $(o_1,o_2,o_3)$ but not
any of its subsets. This effectively compresses smaller patterns.

In addition, when the temporal parameters $K,L,G$ are given,
a larger $M$ actually decreases the number of resulted patterns. 
This is because there are less patterns having more than $M$ objects. 
As shown in Figures 7 (a)(g)(m), a larger $M$ actually reduces the running time of both TRPM and SPARE.
}

\emph{Regarding the second weak point, one line of related work is the superimposition
of constraints on spatiotemporal mining. An example is a road network. In other words, given a road network, the network imposes constraints on GCMP.}

\response{We agree that domain-specific constraints
on spatiotemporal mining could be similar to GCMP.
However, these domain-specific constraints 
are nontrivial to be generalized. Since our GCMP model and technical solutions
do not leverage domain knowledge, we can support
pattern discovery in a broader context. We add a discussion in Section 2.3.
}
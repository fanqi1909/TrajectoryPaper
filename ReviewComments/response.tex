\documentclass{vldb}
\usepackage{url}
\usepackage{amsmath}
\newcounter{ctResp}
\newcommand{\response}[1]{
\addtocounter{ctResp}{1}
\textbf{Response \arabic{ctResp}:} #1 \\}

\begin{document}
\title{Response to reviewers' comments for Paper \# 278}
\maketitle

We are deeply grateful to all the reviewers for the generous and insightful
suggestions which help us to improve this work to a better level.
In the following, we provide explicit response to each raised concern and address our corresponding revisions in the new submission.

\section{Response to Reviewer 1}

\textbf{1.1} \emph{The main con that I've spotted is the fact that the algorithm might have been
more nicely framed also within the Spark environment by taking advantage of
the various possibilities offered by it (e.g. caching of RDDs)}


\textbf{Response:} We agree with the reviewer that utilizing advanced Spark features
would benefit our solutions. Indeed, our SPARE algorithm has already taken advantages of the Spark-only features: (1) caching of RDDs and (2) DAG execution engine, to achieve speedups.

As described in Section 5.3, our SPARE algorithm uses a run-time best-fit strategy to achieve load balance. The best-fit strategy allocates the largest unassigned star to the most empty reducer, which balances the workloads for each reducer. Since the strategy needs to know the sizes of all stars, it takes place after the Map phase and before the Reduce phase.
Leveraging Spark's DAG execution engine, we can easily augment traditional Map-Reduce workflow to the Map-Planning-Reduce workflow. Furthermore, we utilize the RDD caching feature of Spark to store the map result (i.e., RDDs of stars)
during the planning phase and reuse them in the reduce phase.

In Section 6.2.2 Figure 8, we validate the benefits of adopting these Spark features by comparing SPARE with
SPARE-RD. SPARE-RD is identical to SPARE except it uses random allocation rather than best-fit strategy. Benefit from the two Spark features, the SPARE the best-fit strategy
 only takes around 4\% the total execution time. In return, SPARE gains 15\% speedup compared to SPARE-RD.

%As described in Section 5.3, our SPARE algorithm uses a run-time best-fit strategy to achieve load balance. Using the DAG execution engine, we are able to augment traditional Map-Reduce workflow to the Map-Planning-Reduce workflow. With the support of in-memory caching of RDDs,
%the Map results need not to be recomputed after the planning phase. As the result shown in Figure 8,  SPARE with best-fit strategy saves 15\% total time as compared to the randomly partition strategy (i.e., SPARE-RD).  If without Spark's features, such an optimization would be hardly possible, but SPARE-RD is still more efficient than the baseline TRPM.

We do not consider utilizing other Spark extensions such as Spark-GraphX, Spark-Streaming and Spark-MLlib because these extensions, as their name suggest, are for different application scenarios.





\section{Response to Reviewer 1}

\emph{The main con that I've spotted is the fact that the algorithm might have been
more nicely framed also within the Spark environment by taking advantage of
the various possibilities offered by it (e.g. caching of RDDs)}

\response{We agree that utilizing advanced Spark features would 
improve our solutions. In this work, we do not include 
these optimization in-depth because we set out focus 
on solving the challenges in parallelizing the 
GCMP mining algorithms.
We are also in progress of open-sourcing a Spark-tailored
version of SPARE.
} 


\section{Response to Review 2}

\textbf{2.1} \emph{The GCMP generalization is not particularly novel. Putting a maximum gap size
on consecutive segments is well-known
in sequence mining published more than
10 years ago.}

\textbf{Response:}
We admit that the gap parameter is also used in
mining Gap-restricted Sequential Patterns (GSP). 
However, the novelty of this paper lies
on the one-stop solution of mining co-movement patterns.
In modeling, we provide the generalize co-movement pattern to
alleviate the troubles
is not merely on introducing the gap-constraint.
Instead, we provide a one-stop solution for mining co-movement patterns. With
our 

However, the novelty
of this paper is not merely on introducing the gap-constraint to the generalized
co-movement pattern. 
Instead, we provide a one-stop solution (both in modeling and in technical)
for mining various co-movement patterns with flexible constraints.
 

This solution not only alleviate the cumbersome for designing each
individual patterns, but also 

 With
these constraints, users are able to define a more accurate pattern
which are free from the anomalies (e.g., loss-connection anomaly in Figure 2).

Besides, the essence of GSP and GCMP are different. The goal of GSP 
is to find the sequences which conform to the gap constraint and occur frequently.
When counting the occurrence of a sequence in GSP, the object
which a sequence belongs to does not matter. Therefore, 
in a GSP result, the involved objects may be neither
 distinct nor close.
In contrast, GCMP imposes requirements on both the spatial closeness 
of objects and the number of distinct objects.
Due to the unawareness of objects, techniques that are used in GSP 
cannot be directly applied on GCMP. We summarize this difference
in Section 2.5.


\emph{In fact, I have doubts about formulating the GCMP patterns as proposed. Are
we really interested in all sets of movements beyond a cardinality of size $M$? 
Take the Taxi dataset as an example. Let say that there are lots of taxis going
from the airport to downtown. Let say that there are 1000 such taxis. For a
given $M$, are we interested in ${1000 \choose M}$ answers? 
So this speaks to the problem of picking $M$. If $M$ is 500, what is ${1000 \choose M}$? In fact, even if the
system gives the single answer of ${1000\choose 1000}$,
I am not sure I am
interested in this pattern as I already know that there are many taxis going from
the airport to downtown. What I think I am really interested in are GCMP that
are ``anomalous'', which is much harder to define.}

\response{We appreciate the reviewer's concerns on the 
size of the output. 
In fact, both TRPM and SPARE
have considered compressing the output by only discovering 
the patterns with larger object set. 
Particularly,  in the Line Sweeping Mining (Algorithm 1) of TRPM, whenever
a pattern $c$ becomes valid, it is directly outputted (Lines 14-16). This
prohibits outputting $c$'s subset. Similarly, in the Apriori Enumerator (Algorithm 3)
of SPARE, a pattern $c$ is outputted if none of its supersets could become a valid pattern (Line 12-14).
This ensures the output $c$ is not a subset of some pattern.
These mechanisms effectively condense the output by subsuming smaller
patterns using their supersets.
% Nevertheless, these mechanisms are able to preserve ALL valid patterns.
Linking to the Taxi example, if the ground truth contains 1000 taxis travel
together, then both TRPM and SPARE tend to output the single pattern.
In addition, real detection of ``anomalous'' co-moment pattern needs to 
impose stringent constraints via $M,K,L,G$. This would further reduces 
the size of the output.
%On the other hand, in reality, if users wish to detect ``anomalous'' patterns,
%the parameters $M,K,L,G$ will be set to be very stringent. This reduces
%the size of potential outputs. For example, when detecting the Taxi's movements
%without ground truth of 1000 taxis, an input could be $M = 100$, $K=50$ min, $L=30$ min,
%$G = 2$ min. Then, there would be very less patterns satisfying these constraints.
% and the meaningfulness of the output. 
%In terms of the output size, 
%we certainly do not need to output every pattern.
%In fact, both TRPM and SPARE algorithms tend to only output
%the patterns with larger object size. That is to say, for
%the Taxi example, the pattern containing all the 1000 taxis will
%be outputted. To see this, in Algorithm 1, TRPM effectively 
%starts with a temporally invalid pattern and continue
%to shrink its object set until it is temporally valid. Then, the result
%is the ``largest'' valid pattern for these object. Similarly,
%in Algorithm 3, SPARE effectively starts with a temporally valid
%pattern with small object set. Then SPARE continues to grow
%the object set until its temporal dimension is about to be non-decomposable (Definition 5).
%Therefore both TRPM and SPARE tends to find the ``largest'' valid
%patterns.
%
%In terms of the meaningfulness, we believe that all patterns that conformed
%to the parameters need to be outputted. 
%This is because the parameters
%
%
%We see two questions derived from this concern. First, do we need
%to output all ${1000 \choose M}$ patterns if $1000$ taxis form a true GCMP
%pattern? Our answer to this question is no. Therefore, the nature of both
%of our TRPM and SPARE algorithms tend to only output the patterns containing 
%larger sets of objects. To see this, in Algorithm 1, TRPM starts with an temporal invalid
%pattern and continues to shrink its object set while growing its temporal dimension.
%TRPM stops as soon as the object become valid. In an inverse way, in Algorithm 3, 
%SPARE starts with a pattern with small objects and continues to grow its object set
%while reducing its temporal dimension. SPARE stops as soon as the pattern is about
%to be temporally invalid. Therefore, both TRPM and SPARE would prefer to output
%larger pattern.
%
%
%%
%%We agree with the reviewer that if no other parameters are given, the maximum number
%%of patterns discovered could be as large as ${ \mathbb{O} \choose M}$, which may
%%be less interesting to users.
%However, such a case rarely occurs due to two reasons. First,
%in real trajectory mining, parameter $M$
%is associated with temporal parameters $K,L,G$ and they jointly prune many false
%patterns. Second, the nature of our solutions prefer 
%a larger pattern when possible~\footnote{TRPM shrinks an invalid larger pattern until it is valid. SPARE grows a valid smaller pattern until it is about to be invalid. Therefore both algorithms tend to find larger patterns.}.
%For example, if $(o_1,o_2,o_3)$ is 
%a pattern and $M = 2$, 
%then both TRPM and SPARE only output $(o_1,o_2,o_3)$ but not
%any of its subsets. This effectively compresses smaller patterns.
%
%In addition, when the temporal parameters $K,L,G$ are given,
%a larger $M$ actually decreases the number of resulted patterns. 
%This is because there are less patterns having more than $M$ objects. 
%As shown in Figures 7 (a)(g)(m), a larger $M$ actually reduces the running time of both TRPM and SPARE.
}

\emph{Regarding the second weak points, one line of related work is the superimposition
of constraints on spatiotemporal mining. An example is a road network. In other words, given a road network, the network imposes constraints on GCMP.}

\response{We understand that spatiotemporal mining with domain-constraints
looks similar to GCMP.
However, those techniques often heavily utilize the domain knowledge (such as
directions in road networks) which are nontrivial to be generalized.
Differently, since our GCMP model and technical solutions
do not leverage the domain knowledge, we can support
pattern discovery in broader scenarios such as user check-in histories in social networks, visitor movements in a building and 
passenger flows in a city. We also add the discussion in Section 2.3.
}
\section{Responses to Reviewer 3}
\emph{Though the implementation references Spark as the implementation
platform (as it also reflects in github repo provided), the algorithm design is
mostly limited to MapReduce, aka only Hadoop, which is a very small subset of
Spark. This may have a negative impact on the baseline implementation.
Particularly, recent releases of Spark have introduced window functions that can
be applied directly in the sliding window scenario here. Certainly, the algorithm
has to be redesigned to use DataFrame (and/or Spark SQL) interface, it has
been noted that this is a very efficient way to execute window functions in
Spark}

\response{
We thank the reviewer for suggesting a better way of implementing 
the baseline algorithm (i.e., TRPM). However, this could
hardly be done after our investigations. The major reason is
that current version of Spark SQL (i.e., 1.6.2 released in June)
does not support the User Defined Aggregate Function (UDAF) on window
function~\footnote{JIRA Spakr-8641: Native Spark Window Functions \url{https://issues.apache.org/jira/browse/SPARK-8641}}. Without UDAF, implementing
our Line sweep algorithm (Algorithm 1) using Spark SQL 
primitive aggregates (e.g., sum, avg, rank and nth-tile)
is beyond our abilities.

Further, we believe the performance boost on TRPM from Spark SQL
is very limited. The reason is that our 
Line Sweeping Mining algorithm (Algorithm 1) is not a 
properly reducible function. That is, the result of a 
smaller window could not be used to compute the result of a larger
window. In such a case, Spark SQL has to process each
window independently in parallel, which is equivalent to our TRPM implementation.
%We agree that leveraging more advanced Spark features
%could further improve the performance of our solutions. 
%We do not heavily leverage these features because 
%we wish to focus on designing parallel GCMP mining algorithms.
%%to provide a general parallel framework which does not tie to Spark.
%Nevertheless, we are in progress in developing
%a better open-sourced version with Spark-tailored optimization. 
%In the revision, we try to add a DataFrame based TRPM solution 
%which aims to leverage the window functions from Spark-SQL.
%However, to the best of our efforts, we could not achieve this goal in a short time. 
%The major challenge is that current Spark do not support User Defined Aggregate Function (UDAF)
%in window functions~\footnote{JIRA Spakr-8641: Native Spark Window Functions \url{https://issues.apache.org/jira/browse/SPARK-8641}}. Without UDAF, implementing
%our Line sweep algorithm (Algorithm 1) using Spark primitives (e.g., sum, avg, rank and nth-tile)
%is nontrivial. Nonetheless, we expect less performance
%boost from TRPM even with the UDAF support. This is because the Line sweep algorithm is not
%properly reducible, where Spark system would fail to leverage the partial aggregates
%across multiple windows. As a result, it would be equivalent to our current TRPM
%solution where windows are processed independently in parallel.
%TRPM although can be modeled using window functions over data frame, the aggregates
%it uses (i.e., the Line Sweeping method in Algorithm 1) cannot be simply represented
%by Spark primitives (e.g., sum, avg, rank, nth-tile). We then resort to
%the Spark User Defined Aggregate Function (UDAF). However, in Spark 1.5.2, the UDAF
%does not support window function syntax. Even in the latest Spark 1.6.0 (released 
%last month), UDAF does not fit with window function. We confirm these information
%with Spark development team~\footnote{In Spark JIRA tickets, fully supporting UDAF is still under development}.
%
%Nevertheless, since our Algorithm 1 is not reducible, even with UDAF, TRPM could
%not boost too much 
%
%we expect
%even with UDAF support TRPM would not boost too much.  Spark system is 
%%hard to leverage the partial aggregates acrocess
%
%we can analytically expect even using UDAF, TRPM would
%not boost too much. This is because our Algorithm 1 is not reducible. That
%is we cannot merge or share partial aggregates across different windows. In summary,
%We wish to convince the reviewer that current TRPM implementation is reasonable. 
}

\emph{In particular, it would be great to provide the
difference in the number of partitions/splits, the amount of processing and
memory usage (i.e., vcore and memory seconds) between TRPM and SPARE}

\response{We add Table 7 to include vcore-seconds 
and RDD memory usage for both TRPM and SPARE. We also add a clear description on partitions in Section 6.
}


\emph{A plot that breaks down the performance gain by each method would
be greatly appreciated by the readers.}

\response{We complement Figure 8 with the breakdown cost of TRPM. This
comparison showcases the benefits of both Star Partition (in mapshufle phase)
and Apriori Enumeration (in reduce phase) of SPARE.
}

\emph{Some choices of words may need to be reconsidered: for example, "a bunch
of" might not be appropriate in a technical paper.}

\response{We thank the reviewer for the correction. We have changed many unprofessional terms.
}


\emph{References to star partitioning and apriori pruning are missing. Though these
are well known, they need to clearly cited. At least the following reference is
missing:}

\response{We add the corresponding references in Section 5.1 and Section 5.2.}

\emph{In "In contrast, when utilizing the multicore
environment, SPAREP achieves 7 times speedup and SPARES achieves 10 times speedup.", was "multicore"
referring to the use of all 16 cores in one of your node? The specification of the machine was not clear.}

\response{ We use all cores in a single node to conduct the experiments. We revise the description in Section 6.2.3.
} 


\emph{The computation of "eta" was slightly different than that in the paper}

\response{
We have updated the GitHub repository to rectify the typo in the equation.}
\section{Response to Meta Reviewer}

\textbf{M.1} \emph{Novelty: The GCMP generalization is not particularly novel. Please elaborate on the novelty of the work.}

\textbf{Response:} In addition to the points in the response of comments 2.1, another novelty of our work lies in the 
techniques. 
Although the idea of star partition and apriori enumeration are well-known, 
linking them together to solve the problem in trajectory domain has not been attempted before.
Besides, we input heavy details on these algorithms (e.g., theoretical bounds of partition and
anti-monotonicity) which have not been previously proposed.


\textbf{M.2} \emph{Parameter setting: Are we really interested in all sets of movements beyond a cardinality of size M? Please elaborate
on the motivation.}

\textbf{Response:} Addressed in the response for comments 2.2. 

\textbf{M.3} \emph{Spark implementation: the paper references Spark as the implementation platform but the algorithm is limited to MapReduce. Spark has capabilities beyond MapReduce such as window functions. Please provide an implementation that uses the relevant features of Spark, or a convincing discussion of how these features can be useful and why they were not used.}

\textbf{Response:} Addressed in the response for comments 1.1 and 3.1.

\textbf{M.4} \emph{More details in the performance evaluation: Please provide more details about data partitioning and the effect of skew. Also provide details about how star partitioning and a priori pruning contribute to performance. Please provide references to these two methods.}

\textbf{Response:} Addressed in the response for comments 3.2, 3.3, 3.4 and 3.6.


\end{document}
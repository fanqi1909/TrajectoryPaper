\section{Definitions}
\label{sec:definition}
Let $\mathbb{O} = \{o_1 ,o_2,...,o_n\}$ be the set of objects and $\mathbb{T} =(1,2,...,N)$ be the discretized temporal dimension. A time sequence $T$ is defined as an ordered subset of $\mathbb{T}$. Given two time sequences $T_1$ and $T_2$, we define \revised{several} commonly-used operators in this paper in Table~\ref{tbl:operators}.

\begin{table}[h]
\vspace{-2mm}
\centering
\begin{tabular}{|c|l|}
\hline 
\textbf{Operator} & \textbf{Definition} \\ 
\hline
$T[i]$ & the $i$-th element in the sequence $T$ \\ 
\hline
$|T|$ & the number of elements in $T$\\
\hline
$\max(T)$ & the maximum element in $T$\\
\hline
$\min(T)$ & the minimum element in $T$\\
\hline
$\range(T)$ & the range of $T$, i.e., $\max(T) - \min(T) +1$\\ 
\hline 
$T[i:j]$ & subsequence of $T$ from $T[i]$ to $T[j]$ (inclusive) \\ 
\hline
$T_1\subseteq T_2$ &  $\forall T_1[x]\in T_1$, we have $T_1[x]\in T_2$. \\\hline
$T_3=T_1\cup T_2$ & $\forall T_3[x]\in T_3$, we have $T_3[x]\in T_1$ or $T_3[x] \in T_2$\\ 
\hline
$T_3=T_1\cap T_2$ & $\forall T_3[x]\in T_3$, we have $T_3[x]\in T_1$ and $T_3[x] \in T_2$\\ 
\hline
\end{tabular}
\caption{Operators on time sequence.} \label{tbl:operators}
\end{table} 
 
\vspace{-3mm}
We say a sequence $T$ is consecutive 
if $\forall i \in (1,...,|T|-1), T[i+1] = T[i] + 1$.  We refer to each consecutive subsequence of $T$ as a \emph{segment}.
It is obvious that any time sequence $T$ can be decomposed into
segments and we say $T$ is \textit{L-consecutive}~\cite{li2015platoon} 
if the length of every segment is no smaller than $L$. As illustrated in Figure~\ref{fig:platoon_weakpoint}, patterns adopting the notion of $L$-consecutiveness (e.g., \emph{platoon} and \emph{group}) still suffer from the \emph{loose connection} anomaly. 
To avoid such an anomaly without losing generality, we introduce a parameter $G$ to control the gaps between
timestamps in a pattern. Formally, a $G$-connected time sequence is defined as follows:

\begin{definition}[$G$-connected]
A time sequence $T$ is $G$-connected if the gap between any of its neighboring timestamps is no greater than $G$, i.e.,
 $\forall i \in (1,\ldots,|T|-1), T[i+1]-T[i] \leq G$.
\end{definition}

We take $T=(1,2,3,5,6)$ as an example, which can be decomposed into two segments $(1,2,3)$ and $(5,6)$. $T$ is not $3$-consecutive since the length of $(5,6)$ is $2$. Thus, it is safe to say either $T$ is $1$-consecutive or $2$-consecutive. On the other hand, $T$ is $2$-connected since the maximum gap between its neighboring timestamps is $2$. It is worth noting that $T$ is not $1$-connected because the gap between $T[3]$ and $T[4]$ is $2$ (i.e., $5-3=2$).

Given a trajectory database that is discretized into snapshots, we can conduct a clustering method, either disk-based or density-based, to identify groups with spatial proximity. Let $T$ be the set of timestamps in which a group of objects $O$ are clustered. We are ready to define a more general co-movement pattern:
\begin{definition}[General Co-Movement Pattern]
A general co-movement pattern finds a set of objects $O$ satisfying the following five constraints: (1) \textbf{closeness:} the objects in $O$ belong to the same cluster in the timestamps of $T$; (2) \textbf{significance:} $|O| \geq M$; (3) \textbf{duration:} $|T| \geq K$; (4) \textbf{consecutiveness:} $T$ is $L$-consecutive; and (5) \textbf{connection:} $T$ is $G$-connected.
\end{definition}
There are four parameters in our general co-movement pattern, including object constraint $M$ and temporal constraints $K,L,G$. 
%\revised{These parameters have different effects on the number of patterns discovered. Larger $K$, $L$, $M$ 
%put more stringent constraints on patterns such that less patterns could satisfy the requirements. Larger $G$
%relaxes the temporal constraints therefore more patterns could be discovered. Currently, we leave tuning of
%parameters to users to accommodate various application needs.}  
By customizing these parameters, our pattern can 
express other patterns proposed in the literature, as illustrated in Table~\ref{tbl:patterns}. 
In particular, by setting $G=|\mathbb{T}|$, we achieve the \emph{platoon} pattern. By setting $G=|\mathbb{T}|,L=1$, we reach the \emph{swarm} pattern. By setting $G=|\mathbb{T}|$, $M=2$, $K=1$, we gain the \emph{group} pattern. Finally by setting $G=1$, we result in the \emph{convoy} and \emph{flock} patterns. 
In addition to the flexibility of representing other existing patterns, our GCMP is able to avoid the \emph{loose connection} anomaly by tuning the parameter $G$. 
%It is notable that GCMP cannot be modeled by existing patterns. 
\begin{table}[h]
\centering
\begin{tabular}{|l|c|c|c|c|c|}
\hline 
\textbf{Pattern} & $M$ & $K$ & $L$ & $G$ & \textbf{Clustering} \\ 
\hline
Group & $2$ & $1$ & $2$ & $|\mathbb{T}|$ & disk\\
\hline
Flock & $\cdot$ & $\cdot$ & $K$ & $1$ & disk \\
\hline 
Convoy & $\cdot$ & $\cdot$ & $K$ & $1$ & density\\ 
\hline 
Swarm & $\cdot$ & $\cdot$ & $1$ & $|\mathbb{T}|$ & density\\ 
\hline 
Platoon & $\cdot$ & $\cdot$ & $\cdot$ & $|\mathbb{T}|$ & density\\  
\hline 
\end{tabular} 
\caption{Expressing other patterns using GCMP. $\cdot$ indicates a user specified value.
% $M$ represents the  \emph{significance} constraint. $K$ represents the \emph{duration} constraint. $L$ represents the \emph{consecutiveness} constraint. $G$ represents the \emph{connection} constraint.
}
\label{tbl:patterns}
\end{table}

\vspace{-2mm}
Our definition of GCMP is independent of the clustering method. Users can apply different clustering methods to facilitate different application needs. 
We currently expose both disk-region based clustering and DBSCAN as options to the users. In summary, the goal of this paper is to present a parallel solution for discovering all the valid GCMPs from large-scale trajectory databases. Before we move on to the algorithmic part, we list the notations that are used in the following sections.

\begin{table}[h]
\centering
\begin{tabular}{|c|l|} 
\hline
\textbf{Symbol} & \textbf{Meaning} \\
\hline
$S_t$ & snapshot of objects at time $t$ \\
\hline
$M$ & significance constraint \\
\hline 
$K$ & duration constraint\\
\hline
$L$ & consecutiveness constraint\\
\hline
$G$ & connection constraint \\
\hline
$P=\langle O:T \rangle$ & pattern with object set $O$, time sequence $T$\\
\hline
%$C_t(o)$ & cluster of object $o$ at time $t$ \\
%\hline 
$S_t$ & set of clusters at snapshot $t$\\
\hline
$\eta$ & replication factor in the TRPM framework\\
\hline 
$\lambda_t$ & partition with snapshots $S_t,..,S_{t+\eta-1}$ \\
\hline
$G_A$ & aggregated graph in SPARE framework\\
\hline
$Sr_i $ &  star partition for object (vertex) $i$ \\
\hline 
$\Gamma$ & size of the largest star partition\\
\hline
\end{tabular} 
\caption{Summary of the use of notations.}
\end{table}

\appendix
\section{Proofs of Theorems}
\subsection{Proof of Theorem~\ref{THM:RP_ETA}}
%\begin{proof}
%Let $T'$ be the \emph{shortest} valid subsequence (wrt. $K,L,G$) 
%of a valid sequence $T$. Let $\eta$ be the upper bound for all $T'$s among 
%all possible valid sequence. It is easy to see that, under such a setting,
%any valid sequence would be captured by one of the partitions in temporal
%replication. This proves the completeness. Now, we compute the minimum value
%of $\eta$ as follows:
%any $T'$ can be viewed as $n$ consecutive segments with sizes $l_1,..,l_n$
%and $n-1$ gaps with sizes $g_1,...,g_{n-1}$. Since $\eta$ is the upper bond among all $T'$s, $\eta$ can be formulated 
%as follows:
%\begin{equation}
%\eta = \max_{n,l_i,g_i} \{ \Sigma_{i=1}^{i=n} l_i + \Sigma_{i=1}^{i=n-1} g_i \}
%\end{equation}
%With the following constraints: (1)$\forall l_i, L \leq l_i \leq K-1$;(2)
%$\forall g_i, 1 \leq g_i \leq G$; (3) $\Sigma_{i=1}^{i=n} l_i \geq K$ and
%(4) $\Sigma_{i=1}^{i=n-1}l_i  \leq K-1$. The constraint (1)(2)(3) due to the 
%validity of $T'$ and the constraint (4) is because of the minimum size of $T'$.
%Observe that, constraints (1)-(4) form a convex polygon and $\eta$ is monotone
%increasing wrt. $n,l_i,g_i$, the maximum value of $\eta$ is thus taken at the boundaries. Further
%observe that $n \in [2, \lceil \frac{K}{L} \rceil]$ and $K \geq 1$, these
%conditions naturally derive
%$\eta = (\lceil \frac{K}{L} \rceil -1)*G+2K -2$.
%%valid pattern $P$, let $T'$ be the subsequence of $P.T$ which conforms to $K,L,G$ 
%%with the smallest length. Note that there could be many qualified $T'$s. 
%%Let the $i^{th}$ local-consecutive segment of $T'$ be $l_i$ and 
%%let the $i^{th}$ gap of $T'$ be $g_i$. Then, the size of $T'$ can 
%%be written as $\Sigma_i (l_i + g_i)$.  Since $T'$ conforms to $K,L,G$, 
%%then $2K \geq \Sigma_i (l_i) \geq K$, $l_i \geq L$, $g_i \leq G$. 
%%It follows: $\Sigma_i(l_i+g_i) \leq (\lceil \frac{K}{L} \rceil -1) *G+2K$. 
%%If every partition is of at least such a size, then $T'$ must be
%%captured by at least one of the partition. Thus, the pattern $P$ would 
%%be valid in that partition. This proves the completeness.
%\end{proof}

\subsection{Proof of Theorem~\ref{THM:SPM_CORRECT} and Lemma~\ref{LEM:SPM_CORRECT}}


\subsection{Proof of Theorem~\ref{THM:SPM_LB}}
\begin{proof}
Let $B*$ be the assignment matrix wrt the optimal numbering.
Since we have a star for each object, by the degree-sum formula and pigeon-hole theorem, 
$\Gamma^*=||B^*\vec{b}||_\infty \geq d/2$.
Next, given a numbering $P$, let $e_{i,j}$ be an entry in $PAP^T$. Since 
edges in graph $G$ are independent, then $e_{i,j}$s are independent. Moreover,
let $d_i$ denote the degree of vertex $i$, 
then $E[d_i]=E[\Sigma_{1\leq j \leq n}e_{i,j}]=d$. This is because renumbering the
vertexes does not affect the average degree.
Since $B=\text{triu}(PAP^T)$, entries in $B$ can be written as :
\begin{equation*}
b_{i,j} = \begin{cases}
			e_{i,j}, i>j \\
			0, otherwise
		  \end{cases}  
\end{equation*}
There are two observations made. First, since $e_{i,j}$s are independent,
$b_{i,j}$s are independent. Second, since $i>j$ and $e_{i,j}$s are independent. 
$E[b_{i,j}] = E[e_{i,j}]E[i>j]= E[e_{i,j}]/2$.

By definition, $c_i = \Sigma_{1\leq j \leq n} b_{i,j}$, 
is a sum of $n$ independent 0-1 variables. Taking expectation on both sides, 
we get: $E[c_i] = E[\Sigma_{1\leq j \leq n} b_{i,j}]=E[\Sigma_{1\leq j \leq n} e_{i,j}]/2 = d/2$.
 Let $\mu =E[c_i] = d/2$, 
$t = \sqrt{n\log n}$, by Hoeffding's Inequality, the following holds:
\begin{equation*}
\begin{split}
	Pr(c_i \geq \mu + t) 
						&\leq \exp(\frac{-2t^2}{n}) \\
						&= \exp(-2\log n) = n^{-2}
\end{split}
\end{equation*}

The first step is due to the fact that all $b_{i,j}$ are bounded in the range of [0,1]. 
Next, since the event $(\max_{1 \leq j \leq n}(c_j) \geq \mu + t)$ can be viewed as
$\cup_{c_i} (c_i \geq \mu + t )$, by Union Bound, we achieve the following:
\begin{equation*}
\begin{split}
	Pr(\Gamma \geq \mu + t) &=Pr(\max_{1\leq j \leq n}(c_j) \geq \mu + t)  \\
		& = Pr(\cup_{c_i} (c_i \geq \mu + t )) \\
		&\leq \Sigma_{1 \leq i \leq n} Pr(c_i \geq \mu + t) \\
		& = n^{-1} = 1/n
\end{split}
\end{equation*}
Substitute back $t$ and $\mu$, we achieve the following concise form:
\begin{equation*}
	Pr(\Gamma \geq (d/2 + \sqrt{n\log n})) \leq 1/n
\end{equation*}
This indicates that, the probability of $(\Gamma-d/2)$ being less than or equal to $ O(\sqrt{n\log n})$ is $(1-1/n)$. 
With the fact that $\Gamma^* \geq d/2$, we conclude that
with probability greater than $(1-1/n)$, 
the difference between $\Gamma$ and $\Gamma^*$ is less than $O(\sqrt{n\log n})$.
When the aggregated graph is \emph{dense} (i.e., $d\geq \sqrt{12 \log n}$), we may use
the Chernoff Bound instead of Hoeffding's Inequality to derive a tighter bound of 
$O(\sqrt{\log n}) $ with the similar reasoning.
\end{proof}

%\subsection{Proof of Theorem~\ref{THM:SPM_TM}}
%\begin{proof}
%Let $P_1$, $P_2$ be two candidates with $P_1.O \subseteq P_2.O$. It is easy to see that $P_1.T \supseteq P_2.T$.
%Suppose $P_1.T$ cannot be simplified to a candidate sequence. Then
%by proof of contradiction, any subset of $P_1.T$ cannot
%be simplified. It follows that $P_2.T$ cannot be simplified to a candidate sequence. 
%In summary, if $P_1.T$ cannot be simplified, $P_2$ can be pruned. 
%\end{proof}

\subsection{Proof of Theorem~\ref{THM:SPM_FCC}}
\begin{proof}
We prove by contradiction. Suppose there exists another pattern $P$ such that $P.O \neq FC_i$, 
let $X=P.O - FC_i \neq \emptyset$. Consider a subset $P_1$ of $P$ which contains $X$ with size $i+1$, 
(i.e., $P_1 \subseteq P, P_1\subseteq X, |P_1|=i+1$). Since $P$ is a proper pattern, 
then $P_1$ is also a proper pattern. Therefore $P_1 \in Lv_i$.
It follows $X$ is in the forward closure of $FC_i$, (i.e.,$X \in FC_i$), which contradicts with $X\notin FC_i$.
\end{proof}
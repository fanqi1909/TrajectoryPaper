\appendix
%\section{Proofs of Theorems}
\section{Proofs of Theorem~\ref{THM:SPM_LB} and~\ref{THM:SPM_LB_INC}}
\label{apx:thm2proof}
\begin{proof}
$\Gamma$ can be formalized using linear algebra:
Let $J$ be the adjacent matrix of $G_A$.
%Let $G_A$ be an aggregated graph, with a $n \times n$ adjacent matrix $J$.
A vertex order on $G_A$ can be represented as 
%
%Since a vertex order is a permutation of $J$, the adjacent matrices 
%of any reordered graphs can be represented as 
$PJP^T$
where $P$ %$\in \mathbb{P}$ 
is a %$n\times n$ 
\emph{permutation matrix}~\footnote{an identity matrix with rows shuffled}.
Consider an assignment matrix $B$ of star partitioning (i.e., $b_{i,j} = 1$ if vertex $j$ is in star $Sr_i$),
$B=\triu(PJP^T)$~\footnote{\text{triu} is the upper triangle part of a matrix} holds.
% $B=\triu(PJP^T)$~\footnote{\text{triu} is the upper triangle part of a matrix}, then $B$ is the assignment matrix of the star partitioning (i.e., $b_{i,j} = 1$ if vertex $j$ is in star $Sr_i$).
%
%In star partitioning, we assign each edge $e(i,j)$ in $G_A$ to the lower vertex, 
%then the matrix $B=\triu(PJP^T)$~\footnote{\text{triu} is the upper triangle part of a matrix}
%represents the assignment matrix wrt. $P$ (i.e., $b_{i,j} = 1$ if vertex $j$ is in star $Sr_i$).
Let vector $\vec{b}$ be the \textit{one}\footnote{every element in $\vec{b}$ is $1$} 
vector with size $n$. Let $\vec{c} = B\vec{b}$, then each $c_i$ 
denotes the number of edges in star $Sr_i$. Thus, $\Gamma$ can be represented
as the infinity norm of $B\vec{b}$. Let $\Gamma^*$ be the minimum $\Gamma$ among all vertex orderings, that is:
\begin{equation}
\Gamma^* = \min_{P \in \mathbb{P}}{||B\vec{b}||_\infty} \text{ ,where } ||B\vec{b}||_\infty = \max_{1\leq j \leq n}(c_j)
\end{equation}

Let $B^*$ be the optimal assignment matrix. It follows that 
%Since we have a star for each object, by the degree-sum formula and pigeon-hole theorem, 
$\Gamma^*=||B^*\vec{b}||_\infty \geq d/2$.
Next, %for a vertex ordering $P$, 
let $e_{i,j}$ be an entry in $PJP^T$,
% Since  edges in graph $G$ are independent, then
$e_{i,j}$s are independent. Further, $E[\Sigma_{1\leq j \leq n}e_{i,j}]=d$ where $d$ is the average degree of $G_A$.
% Let $d_i$ denote the degree of vertex $i$,
% since a vertex ordering does not
%affect the average degree,
%then $E[d_i]=E[\Sigma_{1\leq j \leq n}e_{i,j}]=d$. 
%Therefore,
On the other hand, $b_{i,j} =e_{i,j}$ for $i > j$.
%
% \begin{cases}
%			e_{i,j}, i>j \\
%			0, otherwise
%		  \end{cases}  $ 
%\vspace{-4mm}
%\begin{equation*}
%\vspace{-1mm}
%b_{i,j} = \begin{cases}
%			e_{i,j}, i>j \\
%			0, otherwise
%		  \end{cases}  
%\end{equation*}
%
%There are two observations. First, 
%%since $e_{i,j}$s are independent,
%$b_{i,j}$s are independent. Second, 
Since $i>j$ and $e_{i,j}$s are independent. 
$E[b_{i,j}] = E[e_{i,j}]E[i>j]= E[e_{i,j}]/2$.
%As $c_i$ is a sum of $n$ independent 0-1 variables ($b_{i.j}$s).
 By linearity 
of expectations,
we get: $E[c_i] = E[\Sigma_{1\leq j \leq n} b_{i,j}]=E[\Sigma_{1\leq j \leq n} e_{i,j}]/2 = d/2$.
 Let $\mu =E[c_i] = d/2$, 
$t = \sqrt{n\log n}$, by Hoeffding Bound, the following holds:
\begin{equation*}
	Pr(c_i \geq \mu + t) \leq \exp(\frac{-2t^2}{n}) = \exp(-2\log n) = n^{-2}
\end{equation*}

%The first step holds since all $b_{i,j}$ are 0-1 variables. 
Next, the event $(\max_{1 \leq j \leq n}(c_j) \geq \mu + t)$ can be viewed as
$\cup_{c_i} (c_i \geq \mu + t )$, by Union Bound, the following holds:
\begin{equation*}
\begin{split}
	Pr(\Gamma \geq \mu + t) &=Pr(\max_{1\leq j \leq n}(c_j) \geq \mu + t) = Pr(\cup_{c_i} (c_i \geq \mu + t )) \\
		&\leq \Sigma_{1 \leq i \leq n} Pr(c_i \geq \mu + t) = n^{-1} = 1/n
\end{split}
\end{equation*}
%Substitute back $t$ and $\mu$, we achieve the following concise form:
%\vspace{-1mm}
%\begin{equation*}
%\small
%\vspace{-1mm}
%	Pr(\Gamma \geq (d/2 + \sqrt{n\log n})) \leq 1/n
%\end{equation*}
This indicates the probability of $(\Gamma-d/2) \leq O(\sqrt{n\log n})$ is $(1-1/n)$. 
Since $\Gamma^* \geq d/2$, Theorem 2 holds.
%it follows with probability greater than $(1-1/n)$, 
%$\Gamma - \Gamma^* \leq O(\sqrt{n\log n})$.
When the aggregated graph is \emph{dense} (i.e., $d\geq \sqrt{12 \log n}$),
the Chernoff Bound can be used to derive a tighter bound of 
$O(\sqrt{d\log n}) $ following a similar reasoning.
\end{proof}

%\subsection{Proof of Theorem~\ref{THM:SPM_TM}}
%\begin{proof}
%Let $P_1$, $P_2$ be two candidates with $P_1.O \subseteq P_2.O$. It is easy to see that $P_1.T \supseteq P_2.T$.
%Suppose $P_1.T$ cannot be simplified to a candidate sequence. Then
%by proof of contradiction, any subset of $P_1.T$ cannot
%be simplified. It follows that $P_2.T$ cannot be simplified to a candidate sequence. 
%In summary, if $P_1.T$ cannot be simplified, $P_2$ can be pruned. 
%\end{proof}

\section{Proof of Theorem~\ref{THM:SPM_CORRECT}}
\label{apx:spm_correct}
\begin{proof}
For soundness, let $P$ be a pattern enumerated by SPARE. For any two objects $o_1, o_2 \in P.O$, the edge $e(o_1,o_2)$ is a superset of $P.T$. 
%By the definition of star, $o_1$ $o_2$ belong to the same cluster at every timestamps in $P.T$. 
As $P.T$ is a valid sequence, by the definition of GCMP, $P$ is a true pattern.
For completeness, let $P$ be a true pattern. Let $s$ be the object with smallest ID in $P.O$. We prove that $P$ must be outputted by Algorithm~\ref{algo:apriori_mining} from $Sr_s$. 
First, based on the definition of star, every object in $P.O$ appears in $Sr_s$. Since $P.T$ is decomposable, then by Lemma 3 %$\forall O' \subseteq O$,
the time sequence of %$O'$ 
any subset would not be eliminated by any $\mathtt{sim}$ operations.  Next, we prove at every iteration \emph{level} $\leq |P.O|$, $P.O \subset O_u$, where $O_u$ is the forward closure. We prove by induction. $level$ = 2 trivially holds. If $P.O \subset O_u$ at \emph{level $i$}, then any subsets of $P.O$ with size $i$ are in the candidate set. %In \emph{level} $i+1$, these subsets grow (in last iteration, they grow to $P.O$). 
This suggests that no subsets are removed by Lines~\ref{code:output1-start}-\ref{code:output2-end}. Then, $P.O \subset U_{i+1}$ holds. Since $P.O$ does not pruned by simplification, monotonicity and forward closure, $P$ must be returned by SPARE.
\end{proof}